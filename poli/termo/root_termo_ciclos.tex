\documentclass[12pt]{article}
\usepackage[latin1]{inputenc}
\usepackage[brazil]{babel}

\title{Termodin�mica - Ciclos}
\author{
        Rafael Sanchez Souza \\
                Engenharia Mecatr�nica\\
        S�o Paulo, \underline{Brazil}            
}
\date{\today}

\begin{document}
\maketitle

\begin{abstract}
Anota��es de Ciclos Termodin�micos
\end{abstract}

\section{Ciclo de Rankine - ou Ciclo de Pot�ncia de Vapor}

\subsection{Hip�teses}
\begin{itemize}
	\item Transfer�ncia de Energia � positiva no sentido hor�rio
	\item Cada componente do ciclo � analizado como um volume de controle.
	\item Processos do fluido de trabalho s�o revers�veis
	\item Turbina e Bomba trabalham adiabaticamente
	\item Efeitos de Energia Potencial e Cin�tica desprez�veis
	\item Vapor Saturado entra na turbina. 
	\item L�quido saturado sai do Condensador \\
\end{itemize}

CICLO IDEAL
\begin{itemize}
	\item Press�o constante na linha
	\item Isentr�pico
\end{itemize}

\subsection{Algumas rela��es}
\paragraph{Turbina}
Vapor a alta temperatura e press�o expande produzindo trabalho
\[\frac{\dot{W_t}}{\dot{m}}=h_1-h_2\]

\paragraph{Condensador}
Transfer�ncia de calor entre vapor e l�quido refrigerante
\[\frac{\dot{Q_{out}}}{\dot{m}}=h_2-h_3\]

\paragraph{Bomba}
O l�quido condensado � bombeado para o Boiler
\[\frac{\dot{W_p}}{\dot{m}}=h_4-h_3\]

Para sistemas revers�veis:
\[\left(\frac{\dot{Q_{in}}}{\dot{m}}\right)_{int,rev}=v_3(p_4-p_3)\]

\paragraph{Boiler}
O l�quido de trabalho completa o ciclo sendo evaporado
\[\frac{\dot{Q_{in}}}{\dot{m}}=h_1-h_4\]

\paragraph{Efici�ncia T�rmica}
\[\eta=\frac{\frac{\dot{W_t}}{\dot{m}}-\frac{\dot{W_p}}{\dot{m}}}{\frac{\dot{Q_{in}}}{\dot{m}}}=1-\frac{\frac{\dot{Q_{out}}}{\dot{m}}}{\frac{\dot{Q_{in}}}{\dot{m}}}=1-\frac{(h_2-h_3)}{(h_1-h_4)}\]

\paragraph{[Van Wylen] Exerc�cio 11.18}
Encontrar:
\begin{itemize}
	\item Trabalho Espec�fico
	\item Transfer�ncia de calor entre componentes
	\item Efici�ncia do Ciclo
\end{itemize}

N�O RESOLVIDO!

\paragraph{[Van Wylen] Exerc�cio 11.24}
Encontrar:
\begin{itemize}
	\item 
	\item 
	\item 
\end{itemize}

\section{Conclusions}\label{conclusions}
We worked hard, and achieved very little.

\bibliographystyle{abbrv}
%\bibliography{simple}

\end{document}
This is never printed
