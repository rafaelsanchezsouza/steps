\documentclass[12pt]{article}
\usepackage[latin1]{inputenc}
\usepackage[brazil]{babel}

\title{CAM - Computer-Aided Manufacturing}
\author{
        Rafael Sanchez Souza \\
                Engenharia Mecatr�nica\\
        S�o Paulo, \underline{Brazil}            
}
\date{\today}

\begin{document}
\maketitle

\begin{abstract}
Guia de estudos de Computer Aided Manufacturing
\end{abstract}

\section{Comando Num�rico}

\begin{itemize}
	\item Elevada Precis�o
	\item Versatilidade
	\item Economia de usinagem
\end{itemize}

\subsection{Prova de CAM 2005}

\paragraph{Quest�o 1}
Interpolador � o circuito eletr�nico que gera sequencias de pulsos necess�rios � movimenta��o da ferramenta.\\

ARQUITETURA INCREMENTAL

Fita Num�rica $->$ Comando Num�rico $->$ Circuito de contagem revers�svel $->$ Atuador $->$ Guias \\

ARQUITETURA ABSOLUTA

Fita Num�rica $->$ Comando Num�rico $->$ Conversor Digital Anal�gico $->$ Circuito Eletr�nico $->$ Atuador $->$ Guias 

\paragraph{Quest�o 3}
O sistema padr�o de coordenadas da m�quina � estabecido com o eixo Z paralelo ao eixo �rvore e eixos X Y paralelos ao movimento das guias.


\section{Previous work}\label{previous work}
A much longer \LaTeXe{} example was written by Gil~\cite{Gil:02}.

\section{Results}\label{results}
In this section we describe the results.

\section{Conclusions}\label{conclusions}
We worked hard, and achieved very little.

\bibliographystyle{abbrv}
\bibliography{simple}

\end{document}
This is never printed
