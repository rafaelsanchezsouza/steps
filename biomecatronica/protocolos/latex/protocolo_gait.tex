%%%%%%%%%%%%%%%%%%%%%%%%%%%%%%%%%%%%%%%%%
% Daily Laboratory Book
% LaTeX Template
%
% This template has been downloaded from:
% http://www.latextemplates.com
%
% Original author:
% Frank Kuster (http://www.ctan.org/tex-archive/macros/latex/contrib/labbook/)
%
% Important note:
% This template requires the labbook.cls file to be in the same directory as the
% .tex file. The labbook.cls file provides the necessary structure to create the
% lab book.
%
% The \lipsum[#] commands throughout this template generate dummy text
% to fill the template out. These commands should all be removed when 
% writing lab book content.
%
% HOW TO USE THIS TEMPLATE 
% Each day in the lab consists of three main things:
%
% 1. LABDAY: The first thing to put is the \labday{} command with a date in 
% curly brackets, this will make a new page and put the date in big letters 
% at the top.
%
% 2. EXPERIMENT: Next you need to specify what experiment(s) you are 
% working on with an \experiment{} command with the experiment shorthand 
% in the curly brackets. The experiment shorthand is defined in the 
% 'DEFINITION OF EXPERIMENTS' section below, this means you can 
% say \experiment{pcr} and the actual text written to the PDF will be what 
% you set the 'pcr' experiment to be. If the experiment is a one off, you can 
% just write it in the bracket without creating a shorthand. Note: if you don't 
% want to have an experiment, just leave this out and it won't be printed.
%
% 3. CONTENT: Following the experiment is the content, i.e. what progress 
% you made on the experiment that day.
%
%%%%%%%%%%%%%%%%%%%%%%%%%%%%%%%%%%%%%%%%%

%----------------------------------------------------------------------------------------
%	PACKAGES AND OTHER DOCUMENT CONFIGURATIONS
%----------------------------------------------------------------------------------------

\documentclass[idxtotoc,hyperref,openany]{labbook} % 'openany' here removes the gap page between days, erase it to restore this gap; 'oneside' can also be added to remove the shift that odd pages have to the right for easier reading

\usepackage[ 
  backref=page,
  pdfpagelabels=true,
  plainpages=false,
  colorlinks=true,
  bookmarks=true,
  pdfview=FitB]{hyperref} % Required for the hyperlinks within the PDF

\usepackage[latin1]{inputenc}  
\usepackage{booktabs} % Required for the top and bottom rules in the table
\usepackage{float} % Required for specifying the exact location of a figure or table
\usepackage{graphicx} % Required for including images
\usepackage{lipsum} % Used for inserting dummy 'Lorem ipsum' text into the template

\newcommand{\HRule}{\rule{\linewidth}{0.5mm}} % Command to make the lines in the title page
\setlength\parindent{0pt} % Removes all indentation from paragraphs

%----------------------------------------------------------------------------------------
%	DEFINITION OF EXPERIMENTS
%----------------------------------------------------------------------------------------

\newexperiment{example}{This is an example experiment}
\newexperiment{example2}{This is another example experiment}
\newexperiment{example3}{This is yet another example experiment}
\newexperiment{table}{This shows a sample table}
%\newexperiment{shorthand}{Description of the experiment}

%---------------------------------------------------------------------------------------

\begin{document}

%----------------------------------------------------------------------------------------
%	TITLE PAGE
%----------------------------------------------------------------------------------------

% \frontmatter % Use Roman numerals for page numbers
% \title{
% \begin{center}
% \HRule \\[0.4cm]
% {\Huge \bfseries Laboratory Journal \\[0.5cm] \Large Master of Science}\\[0.4cm] % Degree
% \HRule \\[1.5cm]
% \end{center}
% }
% \author{\Huge Rafael \\ \\ \LARGE john@smith.com \\[2cm]} % Your name and email address
% \date{Beginning 6 February 2012} % Beginning date
% \maketitle

% \tableofcontents

\mainmatter % Use Arabic numerals for page numbers

%----------------------------------------------------------------------------------------
%	LAB BOOK CONTENTS
%----------------------------------------------------------------------------------------

% Blank template to use for new days:

%\labday{Day, Date Month Year}

%\experiment{}

%Text

%-----------------------------------------

%\experiment{}

%Text

%----------------------------------------------------------------------------------------

%!TEX root=protocolo_gait.tex
\labday{Segunda-Feira, 28 de Julho de 2014}

\experiment{Protocolo de Experimento de Marcha com Metronomo e EMG}

\paragraph{SUBSISTEMAS}
\begin{itemize}
\item Sujeito
\item Metronomo
\item Optitrack (Cameras)
\item EMG
\end{itemize}

	SUJEITO\\
	Sujeito volunt�rio para experi�ncia, nele ser�o instalados alguns sensores\\

	EMG\\
	Equipamento utilizado na analise os impulsos el�tricos enviados pelo c�rebro\\

	METRONOMO\\
	Equipamento utilizado para a determina��o do ritmo de caminhada\\

	MOTION CAPTURE\\
	Sistema de Cameras Infravermelho + Software\\

\paragraph{PREPARACAO DOS SUBSISTEMAS}

	\subparagraph{EMG}
	\begin{itemize}
	\item Roteador
	\item Laptop Bioroblap
	\item Software EMGAnalyzer
	\item BTS FreeEMG
		\begin{itemize}
		\item Palm HP
		\item Eletrodos
		\item External Trigger Box (Subsistema Metronomo)
		\end{itemize}
	\end{itemize}

	\begin{enumerate}
	\item Ligar carregador Eletrodos
	\item Conectar: 
		\begin{enumerate}
		\item Laptop a Roteador (Cabo de rede)
		\item Laptop a External Trigger Box (Cabo USB)
		\end{enumerate}
	\item Ligar Laptop
		\begin{enumerate}
		\item Selecionar Windows 7
		\item Usuario bioroblap Senha bioroblap
		\end{enumerate}
	\item Ligar Software EMGAnalyzer (Atalho na �rea de Trabalho)
	\item Selecionar Trigger por Hardware (Laboratory - Set Trigger) \hfil \\
	Erro: Se a op��o n�o estiver dispon�vel, possivelmente External Trigger Box n�o est� conectada.
		\begin{enumerate}
		\item Selecionar Hardware Trigger
		\item Selecionar Start
		\item ``Save and Exit''
		\end{enumerate}
	\item Palm HP
		\begin{enumerate}
		\item Ligar Palm pela chave pr�xima ao cabo de alimenta��o\hfil \\
		No caso de abrir ambiente Windows, clicar no �cone Windows e em seguida em BTS FreeEMG
		\item Clicar Activate
			\begin{enumerate}
			\item Selecionar Eletrodos Desejados (1 a 8)
			\item Start
			\item Esperar Conectar (Indicacao Verde no Palm e Eletrodos piscando repetidamente)
			\item Ok
			\end{enumerate}
		\item Clicar Remote 
			\begin{enumerate}
			\item Menu
			\item Capture
			\item Arm \hfil \\
			(Opcional) Lock
			\end{enumerate}
		\end{enumerate}
	\item Aquisicao De Sinal
		\begin{enumerate}
		\item New Patient: completar nome, sobrenome e data de nascimento \hfil \\
		(Em caso de paciente j� existente, selecionar paciente na lista � esquerda)
		\item New Session \hfil \\ 
		Nome padrao: session\_01
		\item New Trial \hfil \\
		Sinal sonoro indicar� que tudo est� certo
		\begin{enumerate}
			\item Inserir nome 
			\item Acquire
			\item View
			\item Testar qualidade dos sinais \hfil \\
			OBS: Start Rec n�o est� dispon�vel pois trigger por hardware foi selecionado.
			\end{enumerate}
			\end{enumerate}

	\subparagraph{METRONOMO}
	\begin{itemize}
	\item Circuito distribuidor de sinal
	\item External Trigger Box
	\item Arduino Duemilanove
	\item Alto-Falante
	\end{itemize}	

	\item Ligar Laptop a Arduino - Cabo Cinza (USB Serial Port COM4)
	\item Ligar Fonte do Metronomo \hfil \\
		Voltagem Fonte Fajuta: 6 Volts\\
		Voltagem Real: 12 Volts
	\item Ligar Software do Metronomo (``C:Users/Bioroblap/Desktop/Gait/Metronomo/Program of Metro/Metronome Controller/Bin/Debug/Metronome/Controller.exe'')

	\subparagraph{MOTION CAPTURE}
	\begin{itemize}
	\item 7 C�meras OptiTrack Flex13 + 2 hubs \hfil \\
		Resolu��o: 1280x1024 1.3MP \\
		120 FPS

	\item 1 Marker Wand (varinha com um marcador)
	\item Esquadro Plano de ch�o
	\item Marcadores
	\item Computador Desktop
	\item Software Motive
	\end{itemize}

	\item Verificar se cameras E hubs est�o conectados
	\item Ligar Software Motive (�rea de Trabalho)
*	\item Wizards - Calibration
		\begin{enumerate}
		\item Quick Start: Perform Camera Calibration
		\item Verificar pontos de luminosidade na sala no modo de visualiza��o Tracking
		\item Eliminar todos os pontos que poss�vel fisicamente, com aux�lio de tecidos, etc.
		\item Por fim, no modo Tracking, eliminar virtualmente os pontos de luminosidade restantes clicando em Block Visible
			\begin{itemize}
			\item Para alternar entre visibilidades Tracking e Referencial, clicar nos ``alvos'' referentes a cada camera, no Menu: Group1 (7) (Master) no canto esquerdo do monitor
			\end{itemize}
		\item Montar Wand de tr�s pontos e clicar em Start Wanding. Mover Wand pelo volume de trabalho at� que ``Sufficient for Quality'' = Very High
		\item Calculate
		\item No caso de ``Overall Result'' e ``Overall Quality'' serem Exeptional/Very High clicar em Apply Result ap�s Motive indicar: ``Ready to Apply''
		\item Criar pasta para o teste no formato: Session ``ANO-MES-DIA\_sujeito'' (Ex. ``Session 2014-05-29\_Bernardo'')
		\item Salvar arquivo na pasta referente ao dia com o nome sugerido pelo Motive
		\item Colocar esquadro sobre refer�ncias na esteira
		\item Selecionar Esquadro na imagem 3D virtual
		\item Set Ground Plane
		\item Salvar
		\end{enumerate}
	\item Ligar Metronomo

	\subparagraph{SUJEITO}
	\begin{itemize}
	\item Sujeitos saud�veis entre 18 e 40 anos
	\item Colete Ortop�dico (ORTOCOM Bivalvado)
	\item Esteira (Moviment - LX160 Treadmill)
	\end{itemize}

	\item Imprimir folha para tomada de dados (ou fazer isto virtualmente)
	\item Explicar como ser� realizado o experimento, esclarecer eventuais d�vidas
	\item Apresentar termo de consentimento livre e esclarecido para assinatura
	\item Fixa��o dos marcadores sobre os pontos de interesse (passar algod�o com �lcool, se necess�rio)
	\item Rigid Body: Layout - Create ou CTRL+2
		\begin{enumerate}
		\item Create From Selection
		\item Selecionar 3 pontos vis�veis
		\item Criar 3 Corpos Rigidos: Metron, LF (Left foot) e RF (Right foot)
		\end{enumerate}
	\item Ligar esteira
		\begin{enumerate}
		\item Verificar tomada
		\item Verificar bot�o que liga esteira
		\item Verificar se chave de seguran�a magn�tica est� corretamente afixada
		\end{enumerate}
	\item Com o sujeito sobre a esteira, ajustar lentamente a velocidade at� 4 km/h
	\item Realizar 5 minutos de adapta��o de caminhada sobre a esteira
	\item Realizar 15 sess�es de 3 minutos de caminhada selecionando as devidas velocidades no software
	%\item Ap�s 8 sess�es, retirar colete do sujeito e continuar sess�es
	\item Tomar notas necess�rias
	\item Ao final do teste, realizar question�rio subjetivo
	\item Organizar laborat�rio e equipamentos utilizados
	\end{enumerate}



%-----------------------------------------

\experiment{Observa��es}-------

\end{document}